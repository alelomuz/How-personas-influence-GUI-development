\documentclass[conference]{IEEEtran}
\IEEEoverridecommandlockouts

\usepackage{cite}
\usepackage{amsmath,amssymb,amsfonts}
\usepackage{algorithmic}
\usepackage{graphicx}
\usepackage{textcomp}
\usepackage{xcolor}
% Define new text style command \BibTeX
\def\BibTeX{{\rm B\kern-.05em{\sc i\kern-.025em b}\kern-.08em
    T\kern-.1667em\lower.7ex\hbox{E}\kern-.125emX}}
% Start visible content
\begin{document}

\title{Impact of Demographics and User Personas on GUI Development}

\author{
    \IEEEauthorblockN{1\textsuperscript{st} Hoai Nam Ngo}
    \IEEEauthorblockA{
        \textit{Department of Computer Science} \\
        \textit{Technical University of Munich} \\
        Munich, Germany \\
        ngothommy@gmail.com
    }
    \and
    \IEEEauthorblockN{2\textsuperscript{nd} Klaudia Paździerz}
    \IEEEauthorblockA{
        \textit{Department of Computer Science} \\
        \textit{Technical University of Munich} \\
        Munich, Germany \\
        pazdzierz.klaudia@gmail.com
    }
    \linebreakand
    \IEEEauthorblockN{3\textsuperscript{rd} Nils Hothum}
    \IEEEauthorblockA{
        \textit{Department of Computer Science} \\
        \textit{Technical University of Munich} \\
        Munich, Germany \\
        nils.hothum@tum.de
    }
    \and
    \IEEEauthorblockN{4\textsuperscript{th} Alessandro Lo Muzio}
    \IEEEauthorblockA{
        \textit{Department of Computer Science} \\
        \textit{Technical University of Munich} \\
        Munich, Germany \\
        ge42riy@tum.de
    }
}

% Generate and display the title section
\maketitle

% TODO
% Summarizes the main contribution, methodology, and results in 150–250 words
\begin{abstract}
This paper studies how personas influence the design of graphical user interfaces. We present a survey-based analysis, describe methodology, and discuss implications for software engineers. The results indicate that persona-driven design improves usability and user satisfaction.
Throughout this paper, Netflix serves as a case example to illustrate GUI design decisions.
\end{abstract}

% Keywords which describe the Paper 
\begin{IEEEkeywords}
Graphical User Interface (GUI), User-Centered Design, Personas, User Experience (UX), Software Engineering
\end{IEEEkeywords}

\section{Introduction}
% Importance of high-quality products in a very competitive market
In a market as competitive as the software market, where variables such as geolocation or scalability no longer represent major challenges, focusing on building a truly high-quality product is more crucial than ever before.

% Three main components of a product
Every product, whether software or not, has three main components: \textbf{functionality}, \textbf{aesthetics} and \textbf{usability}. Functionality refers to the features and capabilities that the product offers to its users. Aesthetics refer to the visual appeal and design of the product, including elements such as color schemes, typography, and overall layout. Meanwhile, usability refers to how easy and intuitive it is for users to use the product. Together, these three factors drive user attraction and satisfaction.

% What is a GUI
Two of these elements - aesthetics and practicality - are part of what we call a \textbf{Graphical User Interface} (GUI). GUI development involves more than just designing a pretty interface, but rather creating a bridge between the offered functionality and the user. Therefore, an effective GUI must be both visually appealing and highly intuitive, making it easy for users to accomplish tasks.

% Netflix without GUI -> not successful
Netflix, for example, would likely not be as successful if a user, in order to watch a movie, had to rely on a text-based, black-and-white terminal rather than a clean, well-designed website with menus and icons.

% GUI is user-centered
As its name already implies, a GUI is very \textbf{user-centered}. As always in business, it is all about the user. A GUI must be designed with the end user in mind. The more a GUI is tailored to it's target audience, the more effective it will be.

% Understand the user through demographics and personas
To understand our customer base we use demographics and personas

 allow to understand the user based on a set of traits


to assign to each user a set of attributes that help to better understand the user and his/her needs. While demographics usually refer to statistical data about a population, personas are fictional characters that represent different user types within a targeted demographic.


refer to a multitude of distinct user characteristics that we consider relevant and have an impact on the tailoring of our GUI. The main categories of these characteristics are:

\begin{itemize}
  \item \textbf{User profile data}, e.g. demographics (age, gender, education level), preferences
  \item \textbf{Activity}, e.g. frequency of use, types of tasks performed, and interaction patterns
  \item \textbf{Environment}, e.g. device properties, legal requirements
\end{itemize}


Some of these attributes might be demographic factors (age, gender, education), psychographic characteristics (preferences, cultural background), and cognitive capabilities.


The development of a GUI is a very complex process that involves almost infinitely many variables. 




There are many factors that make of a GUI a great GUI, e.g. modern and consistent design, intuitiveness, quickness of an action, etc.

Most of these factors are quite measurable, e.g. we could do some statistical test.


Understanding the end user is crucial for creating effective, usable and captivating graphical user interfaces. 


\subsection{Subtitle}
Placeholder

% Alessandro
\section{Impact on design}

% Nils
\section{Demographics vs. Personas}
Firstly it is important to distinguish these two terms, since Demographics and user personas capture two complementary perspectives on users in GUI development.

\textbf{Demographics} provide a population-level view that is grounded in observable facts. Attributes such as age, abilities, cultural background, and prior experience reveal systematic differences in perception, motor skills, cognitive load, and typical usage situations\cite{world2008web, rodriguez2022prototype, clemmensen2009cultural, atata2025designing, salminen2021survey}. This evidence is essential for defining baseline requirements for usability and accessibility, for example regarding font sizes, color contrast, input modalities, or supported services. Several guidelines and standards, such as the Web Content Accessibility Guidelines (WCAG) and platform-specific human interface guidelines, explicitly build on such demographic and accessibility considerations to derive concrete design recommendations. However, demographic categories like ``65+'' or ``visually impaired'' remain coarse, as they do not specify concrete goals, everyday tasks, or the actual usage context of a system.

\textbf{Personas} are used to translate such abstract differences into concrete user archetypes that describe typical goals, tasks, and contexts of a target group\cite{gomez2023design}. They help teams design for specific needs instead of an undefined ``average user'' and provide a shared reference for design decisions. However, personas are frequently criticized when they are created without empirical grounding and instead rely on stereotypes or personal assumptions\cite{wagner2014impact}.

Combining both perspectives mitigates these limitations. Demographic evidence can ground personas by ensuring that relevant constraints and user groups are covered, while personas operationalize demographic insights into concrete scenarios and design implications\cite{rodriguez2022prototype,clemmensen2009cultural, howard2015personas}. In this way, demographics define the breadth and diversity of the user population, and personas turn this breadth into actionable guidance for GUI design.

\begin{figure}[t]
  \centering
  \includegraphics[width=0.75\linewidth]{figures/Demographics-vs-personas.png}
  \caption{Demographics provide population-level constraints, while personas capture goals, tasks, and usage context. (Grphic inspired by [TODO])}
  \label{fig:persona-demo}
\end{figure}

\section{Impact of Demographics on GUI Design}
% Alessandro
\subsection{Country- and Region-Specific Requirements}
% Klaudia
\subsection{Age and Life Stage}
% Klaudia
\subsection{Cultural Differences}
% Klaudia
\subsection{Disabilities and Accessibility Needs}

\section{Impact of Personas on GUI Design}
% Thommy
\subsection{Persona-Driven Adaptation to Context (Mobile vs.\ Desktop)}
% Thommy
\subsection{Limitations and Challenges of Persona Use}

% Nils
\section{Impact of Artificial Intelligence in GUI Development}

% Thommy
\section{User Experience}
% Alessandro
\section{Conclusion}
Section reference \ref{AA}

\section*{Acknowledgment}
We would like to thank Sidong Feng for supervising this project and providing valuable guidance on Graphical User Interface design. We also thank the Chair of Software Engineering \& AI at TUM for providing course resources and support.

\bibliographystyle{IEEEtran}
\bibliography{references}

\begin{comment}
TODO:
- Vibe coding
- Impact of user-tailored GUI on the code
\end{comment}
\end{document}
