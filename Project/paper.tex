\documentclass[conference]{IEEEtran}
\IEEEoverridecommandlockouts

\usepackage{cite}
\usepackage{amsmath,amssymb,amsfonts}
\usepackage{algorithmic}
\usepackage{graphicx}
\usepackage{textcomp}
\usepackage{xcolor}
% Define new text style command \BibTeX
\def\BibTeX{{\rm B\kern-.05em{\sc i\kern-.025em b}\kern-.08em
    T\kern-.1667em\lower.7ex\hbox{E}\kern-.125emX}}
% Start visible content
\begin{document}

\title{Impact of Demographics and User Personas on GUI Development}

\author{
    \IEEEauthorblockN{1\textsuperscript{st} Hoai Nam Ngo}
    \IEEEauthorblockA{
        \textit{Department of Computer Science} \\
        \textit{Technical University of Munich} \\
        Munich, Germany \\
        ngothommy@gmail.com
    }
    \and
    \IEEEauthorblockN{2\textsuperscript{nd} Klaudia Paździerz}
    \IEEEauthorblockA{
        \textit{Department of Computer Science} \\
        \textit{Technical University of Munich} \\
        Munich, Germany \\
        pazdzierz.klaudia@gmail.com
    }
    \and
    \IEEEauthorblockN{3\textsuperscript{rd} Nils Hothum}
    \IEEEauthorblockA{
        \textit{Department of Computer Science} \\
        \textit{Technical University of Munich} \\
        Munich, Germany \\
        nils.hothum@tum.de
    }
    \and
    \IEEEauthorblockN{4\textsuperscript{th} Alessandro Lo Muzio}
    \IEEEauthorblockA{
        \textit{Department of Computer Science} \\
        \textit{Technical University of Munich} \\
        Munich, Germany \\
        ge42riy@tum.de
    }
}

% Generate and display the title section
\maketitle

% TODO
% Summarizes the main contribution, methodology, and results in 150–250 words
\begin{abstract}
This paper studies how personas influence the design of graphical user interfaces. We present a survey-based analysis, describe methodology, and discuss implications for software engineers. The results indicate that persona-driven design improves usability and user satisfaction.
Throughout this paper, Netflix serves as a case example to illustrate GUI design decisions.
\end{abstract}

% Keywords which describe the Paper 
\begin{IEEEkeywords}
Graphical User Interface (GUI), User-Centered Design, Personas, User Experience (UX), Software Engineering
\end{IEEEkeywords}

\section{Introduction}
% Importance of high-quality products in a very competitive market
In a market as competitive as the software market, where variables such as geolocation or scalability no longer represent major challenges, focusing on building a truly high-quality product is more crucial than ever before.

% Three main components of a product
Every product, whether software or not, has three main components: \textbf{functionality}, \textbf{aesthetics} and \textbf{usability}. Functionality refers to the features and capabilities that the product offers to its users. Aesthetics refer to the visual appeal and design of the product, including elements such as color schemes, typography, and overall layout. Meanwhile, usability refers to how easy and intuitive it is for users to use the product. Together, these three factors drive user attraction and satisfaction.

% What is a GUI
Two of these elements - aesthetics and practicality - are part of what we call a \textbf{Graphical User Interface} (GUI). GUI development involves more than just designing a pretty interface, but rather creating a bridge between the offered functionality and the user. Therefore, an effective GUI must be both visually appealing and highly intuitive, making it easy for users to accomplish tasks.

% Netflix without GUI -> not successful
Netflix, for example, would likely not be as successful if a user, in order to watch a movie, had to rely on a text-based, black-and-white terminal rather than a clean, well-designed website with menus and icons.

% GUI is user-centered
As its name already implies, a GUI is very \textbf{user-centered}. As always in business, it is all about the user. A GUI must be designed with the end user in mind. The more a GUI is tailored to it's target audience, the more effective it will be.

% Understand the user through demographics and personas
To understand our customer base we use demographics and personas

 allow to understand the user based on a set of traits


to assign to each user a set of attributes that help to better understand the user and his/her needs. While demographics usually refer to statistical data about a population, personas are fictional characters that represent different user types within a targeted demographic.


refer to a multitude of distinct user characteristics that we consider relevant and have an impact on the tailoring of our GUI. The main categories of these characteristics are:

\begin{itemize}
  \item \textbf{User profile data}, e.g. demographics (age, gender, education level), preferences
  \item \textbf{Activity}, e.g. frequency of use, types of tasks performed, and interaction patterns
  \item \textbf{Environment}, e.g. device properties, legal requirements
\end{itemize}


Some of these attributes might be demographic factors (age, gender, education), psychographic characteristics (preferences, cultural background), and cognitive capabilities.


The development of a GUI is a very complex process that involves almost infinitely many variables. 




There are many factors that make of a GUI a great GUI, e.g. modern and consistent design, intuitiveness, quickness of an action, etc.

Most of these factors are quite measurable, e.g. we could do some statistical test.


Understanding the end user is crucial for creating effective, usable and captivating graphical user interfaces. 


\subsection{Subtitle}
Placeholder

\section{Demographics vs. Personas}
Section reference \ref{AA}

\section{Impact of Demographics on GUI Design}
\subsection{Country- and Region-Specific Requirements}
\subsection{Age and Life Stage}
\subsection{Cultural Differences}
\subsection{Disabilities and Accessibility Needs}

\section{Impact of Personas on GUI Design}
\subsection{Persona-Driven Adaptation to Context (Mobile vs.\ Desktop)}
\subsection{Limitations and Challenges of Persona Use}

\section{Impact of Artificial Intelligence in GUI Development}

% Thommy
\section{User Experience}

\section{Conclusion}
Section reference \ref{AA}

\subsection{Subtitle 1}\label{AA}
Placeholder

An excellent style manual for science writers is \cite{Smith2020}.

\section*{Acknowledgment}
We would like to thank Sidong Feng for supervising this project and providing valuable guidance on Graphical User Interface design. We also thank the Chair of Software Engineering \& AI at TUM for providing course resources and support.

\bibliographystyle{IEEEtran}
\bibliography{references}

\end{document}




\begin{comment}
TODO:
- Vibe coding
- Impact of user-tailored GUI on the code
\end{comment}
