% Source: https://cog2025.inesc-id.pt/call-for-papers/ | Full papers have an 8-page limit in IEEE 2-column format (including references and appendices) and should constitute a technical or empirical contribution to scientific, technical, or engineering aspects of games.

\documentclass[conference]{IEEEtran}
\usepackage{float}
\IEEEoverridecommandlockouts
% The preceding line is only needed to identify funding in the first footnote. If that is unneeded, please comment it out.
\usepackage{amsmath,amssymb,amsfonts}
\usepackage{todonotes}
\usepackage{algorithmic}
\usepackage[american]{babel}
\usepackage{graphicx}
\usepackage{textcomp}
\usepackage{xcolor}
\usepackage{subcaption}
\usepackage[%
  backend=biber,
  url=false,
  style=numeric,
  maxnames=4,
  minnames=3,
  maxbibnames=150,
  giveninits,
  uniquename=init]{biblatex}
\usepackage[hidelinks]{hyperref} % hidelinks removes colored boxes around references and links

\renewcommand*{\thefootnote}{\fnsymbol{footnote}}

\makeatletter % changes the catcode of @ to 11
\newcommand{\linebreakand}{%
  \end{@IEEEauthorhalign}
  \hfill\mbox{}\par
  \mbox{}\hfill\begin{@IEEEauthorhalign}
}
\makeatother % changes the catcode of @ back to 12

\addto\extrasamerican{
	\def\lstnumberautorefname{Line}
	\def\chapterautorefname{Chapter}
	\def\sectionautorefname{Section}
	\def\subsectionautorefname{Subsection}
	\def\subsubsectionautorefname{Subsubsection}
}

\def\BibTeX{{\rm B\kern-.05em{\sc i\kern-.025em b}\kern-.08em
    T\kern-.1667em\lower.7ex\hbox{E}\kern-.125emX}}
\bibliography{bibliography}
\begin{document}

\title{Impact of Demographics and User Personas on GUI Development}

\author{
 \IEEEauthorblockN{1\textsuperscript{st} Hoai Nam Ngo}
    \IEEEauthorblockA{
        \textit{Department of Computer Science} \\
        \textit{Technical University of Munich} \\
        Munich, Germany \\
        ngothommy@gmail.com
    }
    \and
    \IEEEauthorblockN{2\textsuperscript{nd} Klaudia Paździerz}
    \IEEEauthorblockA{
        \textit{Department of Computer Science} \\
        \textit{Technical University of Munich} \\
        Munich, Germany \\
        pazdzierz.klaudia@gmail.com
    }
    \linebreakand
    \IEEEauthorblockN{3\textsuperscript{rd} Nils Hothum}
    \IEEEauthorblockA{
        \textit{Department of Computer Science} \\
        \textit{Technical University of Munich} \\
        Munich, Germany \\
        nils.hothum@tum.de
    }
    \and
    \IEEEauthorblockN{4\textsuperscript{th} Alessandro Lo Muzio}
    \IEEEauthorblockA{
        \textit{Department of Computer Science} \\
        \textit{Technical University of Munich} \\
        Munich, Germany \\
        ge42riy@tum.de
    }
}

\maketitle

\begin{abstract}
This paper presents a structured literature review on how user personas and demographic factors influence graphical user interface (GUI) development. We systematically analyze existing research on persona-based design, demographic differences in user populations (e.g., age, expertise, and cultural background), and their reported impact on usability and user experience. The reviewed work shows that personas are widely used as tools for communication and design rationale, yet only a subset of studies empirically evaluates their contribution to improved GUI quality. Demographic factors are frequently acknowledged but are often treated superficially or inconsistently operationalized across studies. Based on our synthesis, we propose a conceptual framework that links personas, demographics, and GUI design decisions, identify gaps in current research (such as limited validation of personas and narrow demographic sampling), and outline implications and directions for future work in persona-driven, demographically aware GUI development.
\end{abstract}

\begin{IEEEkeywords}
Graphical User Interface (GUI) · User-Centered Design · Personas · User Experience (UX) · Software Engineering
\end{IEEEkeywords}

% !TeX root = ../main.tex

\section{Introduction}
With the growth of computers and the programs that accompany people in their everyday lives, the demands and expectations placed on graphical user interfaces (GUIs) have also increased. GUIs are the most visible interface between humans and software and strongly shape how intuitive and usable a system feels \cite{rodriguez2022prototype}. However, GUIs are also one of the most common places where systems unintentionally exclude user groups, especially when accessibility constraints are not considered early \cite{world2008web}. Many interfaces are designed for the ``standard user'' in the hope of covering the optimum for as large a group of people as possible, even though actual use is characterized by large differences in age, abilities, cultural context, and experience \cite{clemmensen2009cultural,atata2025designing}.

This demographic diversity affects interaction efficiency, error frequency, and cognitive load, and it can determine whether a system is accessible at all \cite{wagner2014impact,gomez2023design}. To address this complexity, development teams often structure user needs using personas. Against this background, this literature review examines what evidence research provides on the influence of demographic factors on GUI design, how personas are conceptualized and evaluated, and which approaches are suitable for consistently combining both perspectives in GUI development.

\subsection{Complementary Roles of Demographics and Personas in GUI Development}
Demographic characteristics and personas address two complementary perspectives on users. Demographic data primarily provide insights at the population level: they show how restrictions (e.g., perception, motor skills) and needs differ between user groups and motivate minimum requirements for accessibility and usability \cite{world2008web,clemmensen2009cultural}. This view provides important context, but it is often too broad to guide specific design decisions. Categories such as ``65+'' or ``visual impairment'' do not explain which goals dominate, which tasks are relevant in everyday life, or in which context an interface is actually used.

Personas are commonly used to translate abstract differences into user archetypes that describe typical \textbf{goals, tasks} and \textbf{usage contexts} of a target group \cite{pruitt2003personas}. They help development teams align design decisions with specific needs rather than focusing on the supposed ``average user.'' At the same time, personas are criticized when they are created without an empirical basis and instead rely on stereotypes or personal assumptions \cite{howard2015personas}. Combining both approaches reduces these weaknesses: demographic evidence can ground personas (e.g., ensuring coverage of relevant constraints), while personas operationalize demographic insights for concrete design decisions \cite{salminen2021survey}.

\begin{figure}[t]
  \centering
  \includegraphics[width=0.75\linewidth]{figures/Demographics-vs-personas.png}
  \caption{Demographics provide population-level constraints, while personas capture goals, tasks, and usage context.}
  \label{fig:persona-demo}
\end{figure}

\subsection{Research Questions}
\begin{itemize}
    \item \textbf{RQ 1:} Which demographic factors are associated with UI/usability effects in the literature, and what GUI implications are derived from this?
    \item \textbf{RQ 2:} What types of personas (classic, data-driven, inclusive/persona spectrum) are described in the literature, and what evidence exists for their benefits and limitations?
    \item \textbf{RQ 3:} What approaches do studies/guidelines suggest for consistently integrating demographic findings and personas into GUI decisions?
\end{itemize}

\section{Review Methodology}
\label{sec:methodology}

This paper presents a structured literature review rather than an empirical user study. The goal of the review is to synthesize how existing research addresses (i) the use of user personas in graphical user interface (GUI) development and (ii) the role of demographic factors in GUI and user experience (UX) design. To ensure transparency and reproducibility, we follow a simple but systematic process consisting of four main steps: search, selection, data extraction, and qualitative synthesis.

\subsection{Search Strategy}

We conducted a database search in major digital libraries commonly used in human--computer interaction and software engineering research, namely IEEE Xplore, the ACM Digital Library, and SpringerLink. In addition, we performed a complementary search on Google Scholar to capture relevant work that may not be indexed uniformly across these venues.

The search focused on combinations of terms related to personas, demographics, and user interfaces. Example search strings included:

\begin{itemize}
  \item \textit{("user persona" OR "personas") AND ("user interface" OR "GUI" OR "UX")}
  \item \textit{demographics AND ("usability" OR "user experience" OR "user interface")}
  \item \textit{persona-based design AND GUI}
\end{itemize}

We restricted our search to peer-reviewed publications written in English and published between 2000 and 2025. The lower bound reflects the period in which personas became widely discussed in interaction design and software engineering, while the upper bound corresponds to the time of conducting this review.

\subsection{Inclusion and Exclusion Criteria}

To keep the scope focused on GUI-relevant work, we applied the following inclusion criteria:

\begin{itemize}
  \item The paper is peer-reviewed (journal, conference, or workshop).
  \item The work addresses user interfaces, graphical user interfaces, or interactive systems with a clear human-facing component.
  \item The paper either (a) uses or discusses user personas in the context of design or evaluation, or (b) explicitly considers demographic characteristics (e.g., age, gender, expertise, cultural background) in relation to usability or UX.
\end{itemize}

We excluded papers that met one or more of the following conditions:

\begin{itemize}
  \item No clear relation to user interfaces or human--computer interaction (e.g., purely backend or infrastructure-oriented work).
  \item Marketing and business-oriented persona usage without any link to interface design or UX.
  \item Non-scientific sources such as blog posts, white papers without peer review, or theses not published in curated venues.
\end{itemize}

These criteria are intentionally simple but sufficient to filter out work that is clearly outside the scope of GUI development, while retaining a broad view on personas and demographics.

\subsection{Study Selection Process}

The database queries returned an initial set of publications. We first removed duplicates across the different digital libraries. Next, we conducted a title and abstract screening to exclude obviously irrelevant papers (e.g., pure marketing analytics or demographic surveys without any interface component).

For the remaining candidates, we inspected the full text to verify that the work (i) addressed user interfaces or interactive systems and (ii) either applied personas or analyzed demographic factors in a way that is relevant to GUI or UX design. This process resulted in a final set of studies that form the basis of our review. While we do not claim full exhaustiveness, the selection procedure aims to cover the central streams of research at the intersection of personas, demographics, and GUI development.

\subsection{Data Extraction and Synthesis}

For each included paper, we extracted a small set of descriptive and analytical attributes:

\begin{itemize}
  \item bibliographic information (authors, year, venue);
  \item application domain (e.g., web, mobile, games, enterprise systems);
  \item role of personas (e.g., requirements elicitation, design rationale, communication, evaluation);
  \item demographic factors considered (e.g., age, gender, expertise, culture, accessibility needs);
  \item reported outcomes related to usability, user experience, or user performance;
  \item study type (e.g., controlled experiment, case study, survey, conceptual or framework paper).
\end{itemize}

Rather than performing a quantitative meta-analysis, we conducted a qualitative, thematic synthesis. Specifically, we grouped papers according to recurring themes such as how personas are conceptualized and used in practice, which demographic variables are most frequently considered, and how these aspects are reported to influence GUI design decisions and UX outcomes. The thematic categories emerging from this synthesis form the structure of the findings presented in Section~\ref{sec:findings}.

\input{03_Thematic Background: Personas and Demographics in GUI Design}
\section{Findings from the Literature}

\subsection{Overview of Included Studies}

\subsection{Roles of Personas in GUI Development}

\subsection{Demographic Factors Considered in GUI/UX Studies}

\subsection{Reported Impact on Usability and User Experience}

\subsection{Design Patterns and Recurring Themes}
\section{Discussion}

\subsection{Answers to the Research Questions}

\subsection{Gaps and Limitations in Existing Literature}

\subsection{Implications for GUI Designers and Software Engineers}

\subsection{Limitations of This Review}

\section*{Acknowledgment}
We would like to thank Sidong Feng for supervising this project and providing valuable guidance on GUI design. We also thank the Chair of Software Engineering \& AI at TUM for providing course resources and support.


\printbibliography

\end{document}
