\subsection{AI-Assisted Development for Persona- and Demographic-Aware GUIs}
\begin{itemize}
  \item \textbf{Embedding personas and demographics into prompts:}
  \begin{itemize}
    \item AI coding assistants can be guided with explicit information about target personas (e.g., novice vs.\ expert users) and demographic constraints (e.g., older adults, users with visual impairments).
    \item When prompts include these aspects, generated GUI code is more likely to reflect appropriate interaction complexity, font sizes, contrast, or navigation depth for the intended user groups.
  \end{itemize}
  \item \textbf{Operationalizing persona and demographic requirements in code:}
  \begin{itemize}
    \item AI can help translate high-level persona descriptions (goals, tasks, context) into concrete UI elements and interaction flows (e.g., simplified wizards for infrequent users, shortcut-heavy views for experts).
    \item Demographic evidence (e.g., accessibility guidelines, age-related constraints) can be referenced explicitly in prompts so that the assistant proposes components that follow these constraints.
  \end{itemize}
  \item \textbf{Risk of generic, persona-blind suggestions:}
  \begin{itemize}
    \item Without explicit persona and demographic context, AI defaults to generic patterns that may primarily fit ``average'' or highly experienced users.
    \item This can unintentionally marginalize critical user segments, such as older users or people with disabilities, even if personas for these groups exist in the project.
  \end{itemize}
\end{itemize}

\subsection{Vibe Coding and the Role of Personas and Demographics}
\begin{itemize}
  \item \textbf{Vibe coding with persona-aware prompts:}
  \begin{itemize}
    \item During exploratory ``vibe coding'', developers can incorporate persona labels (e.g., ``design this view for an anxious first-time user'') and demographic cues (e.g., ``suitable for low-vision users on mobile'') directly in natural-language prompts.
    \item This keeps rapid prototyping aligned with the defined user models instead of drifting towards purely aesthetic or developer-centric preferences.
  \end{itemize}
  \item \textbf{Danger of drifting away from user models:}
  \begin{itemize}
    \item If vibe coding relies only on generic prompts like ``modern dashboard'' or ``clean settings page'', AI-generated GUIs may ignore the specific needs of the personas and demographic groups identified earlier in the project.
    \item Such drift makes it harder to justify design decisions with respect to user diversity and may contradict accessibility and inclusion goals.
  \end{itemize}
\end{itemize}

\subsection{Limitations of AI for Persona- and Demographic-Sensitive GUIs}
\begin{itemize}
  \item \textbf{Lack of embedded user models:}
  \begin{itemize}
    \item Current AI tools do not maintain an internal, project-specific model of personas or demographic segments; they only react to what is stated in the prompt or visible in the code.
    \item As a result, persona and demographic considerations can easily be lost once they are not explicitly mentioned in each interaction.
  \end{itemize}
  \item \textbf{Inconsistent alignment with guidelines:}
  \begin{itemize}
    \item Even when accessibility or demographic constraints are mentioned, AI suggestions may only partially follow relevant guidelines (e.g., WCAG) and still require manual review against persona and demographic requirements.
    \item Over-reliance on AI-generated code increases the risk that subtle needs of certain personas (e.g., low digital literacy, anxiety, motor limitations) are overlooked.
  \end{itemize}
\end{itemize}