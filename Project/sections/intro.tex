% Importance of high-quality products in a very competitive market
In a market as competitive as the software market, where variables such as geolocation or scalability no longer represent major challenges, focusing on building a truly high-quality product is more crucial than ever before.

% What is a GUI
Every consumer product, whether software or not, has three main components: \textbf{functionality}, \textbf{aesthetics} and \textbf{usability}. Together, these three factors drive user attraction and satisfaction. Software development consists of a wide range of often-overlapping activities, each with its own name and scope, with the common goal of delivering a very functional, aesthetic and usable product. One of these activities is the \textbf{Graphical User Interface development}, which focuses on the user interaction aspects by curating the aesthetics and usability parts of a software. GUI development is more than just designing a pretty interface; it spans throughout activities such as requirements analysis and visual design, up to implementation, testing and evaluation. An effective GUI represents a bridge between the functionality and the user and must be both visually appealing and highly intuitive, making it easy for users to accomplish tasks.

% Netflix without GUI -> not successful
Netflix, for example, would likely not be as successful if a user, in order to watch a movie, had to rely on a text-based, black-and-white terminal rather than a clean, well-designed website with menus and icons.

% GUI is user-centered
As its name already implies, a GUI is very \textbf{user-centered}. As always in business, it is all about the user. A GUI must be designed and developed with the end user in mind. The more a GUI is tailored to it's target audience, the more effective it will be.

% Understand the user through demographics and personas
To personalize and tailor a GUI, it is crucial to understand who the end user is. Demographics and user personas are two powerful tools that allow to categorize and understand users on a deeper level. Demographics describe population-level statistics (e.g., age, gender, education, location), whereas personas are crafted fictional characters that embody typical users' goals, behaviours, motivations, and skill levels. Both concepts must be data-backed, behaviorally meaningful, and continuously revised to ensure they accurately reflect the target audience. Coming up with accurate demographics and personas is usually a three-step process:
\begin{itemize}
  \item \textbf{Collect data}, e.g. through surveys, interviews, and analytics
  \item \textbf{Create personas based on analyzed data}
  \item \textbf{Validate and refine personas over time}
\end{itemize}

% Impact on GUI development
Once the demographics and personas are defined, they can be used to improve the GUI development process in three complementary, not mutually exclusive categories:
\begin{itemize}
  \item \textbf{Population-Level Adaptation},
  \item \textbf{Individual-Level Adaptation} and
  \item \textbf{Process \& Validation Influence}.
\end{itemize}