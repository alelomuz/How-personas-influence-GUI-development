\subsection{Graphical User Interface Development}
GUI development encompasses the design, implementation and evaluation of all visual and interactive components through which users interact with a system.
In contrast to simple command-line interfaces, GUIs relay on graphical elements such as windows, buttons, icons, text fields and form elements that mediate user input and system feedback.
At its core, GUI development seeks to align all components with the user's perceptual, cognitive, and motor capabilities, as well as with their goals and tasks in the given context of the use of the product\cite{shneiderman2010designing}.

In the context of softwere systems and for the scope of this paper, it is usefull to distinguish between the terms \emph{GUI design} and \emph{GUI development}.
The focus of GUI design are on the conceptual and visual aspects of an interface: information architecture, interaction flows, layout, typhography, colours and overall look-and-feel.
Its goal is to define \emph{what} the interface should communicate and \emph{how} interaction should feel from a user’s perspective\cite{henderson2002interaction}.

GUI development, in contrast, comprises the engineering activities required to turn these design artefacts into a robust, interactive, and maintainable software component.
It focuses on \emph{how} the specified interface is realised in code, how it behaves under different conditions, and how it integrates with the underlying application logic and data sources\cite{fowler2012patterns}.

From an engineering perspective, GUI development typically involves (i) translating user and system requirements into concrete UI behaviour and constraints, (ii) selecting suitable frameworks and architectural patterns (e.g., MVC or MVVM) \cite{fowler2012patterns},
(iii) implementing components, event handling, state management, and validation, and (iv) assuring quality through testing, performance optimisation, and accessibility checks.
These steps are usually executed iteratively based on feedback from usage data and evaluation.

Importantly, GUI development is not a purely technical exercise.
Implementation decisions about layout structure, interaction complexity, feedback timing, and support for assistive technologies implicitly encode assumptions about the intended user population and the standart for these\cite{world2008web}.
If these assumptions do not match the actual user base, the resulting interface may be correct from an implementation standpoint but still fail in terms of usability and accessibility.

In the following, we focus on how characteristics of the target user group inform GUI development, not directly on the Design components.
We first distinguish demographics from user personas and discuss how both perspectives can be combined to derive concrete implications for graphical user interfaces.

\subsection{User Demographics vs.\ Personas}
Firstly it is important to distinguish these two terms, since Demographics and user personas capture two complementary perspectives on users in GUI development.

\textbf{Demographics} provide a population-level view that is grounded in observable facts. Attributes such as age, abilities, cultural background, and prior experience reveal systematic differences in perception, motor skills, cognitive load, and typical usage situations\cite{world2008web, rodriguez2022prototype, clemmensen2009cultural, atata2025designing, salminen2021survey}. This evidence is essential for defining baseline requirements for usability and accessibility, for example regarding font sizes, color contrast, input modalities, or supported services. Several guidelines and standards, such as the Web Content Accessibility Guidelines (WCAG)\cite{world2008web
} and platform-specific human interface guidelines, explicitly build on such demographic and accessibility considerations to derive concrete design recommendations. However, demographic categories like ``65+'' or ``visually impaired'' remain coarse, as they do not specify concrete goals, everyday tasks, or the actual usage context of a system.

\textbf{Personas} are used to translate such abstract differences into concrete user archetypes that describe typical goals, tasks, and contexts of a target group\cite{gomez2023design}. They help teams design for specific needs instead of an undefined ``average user'' and provide a shared reference for design decisions. However, personas are also criticized when they are created without empirical grounding and instead rely on stereotypes or personal assumptions\cite{pruitt2003personas}.

Combining both perspectives mitigates these limitations. Demographic evidence can ground personas by ensuring that relevant constraints and user groups are covered, while personas operationalize demographic insights into concrete scenarios and design implications\cite{rodriguez2022prototype,clemmensen2009cultural}. In this way, demographics define the breadth and diversity of the user population, and personas turn this breadth into actionable guidance for GUI design.

\begin{figure}[t]
  \centering
  \includegraphics[width=0.75\linewidth]{figures/Demographics-vs-personas.png}
  \caption{Demographics provide population-level constraints, while personas capture goals, tasks, and usage context.}
  \label{fig:persona-demo}
\end{figure}