\subsection{What is the Design Process?}
The development of a modern system or a product is complex and requires a lot of planning. A design process lays the foundation for such planning by categorizing the development into the five most essential stages [1]: 
\begin{itemize}
  \item \textbf{Emphasize }, This stage involves getting data about the userbase and target audience. To successfully build a system or product, one needs to know who and how they will interact with the project.  
  \item \textbf{Define }, The next step is to define the problem that needs to be solved by narrowing down the data to get the most common problems. 
  \item \textbf{Ideate }, Once the problems are defined, thinking about possible solutions and gathering ideas is the next stage in development.  
  \item \textbf{Prototype  }, Next is designing a mockup prototype to visualize the idea; it doesn’t need to be functional yet.
  \item \textbf{Prototype  }, The final stage is testing with stakeholders or test users to validate your ideas.  
\end{itemize}

\subsection{Requirement and Software Engineering }
The stages of the design process can be separated into two aspects of software development: Software Engineering and Requirement Engineering.  

Software Engineering specializes in developing the technical part of the process. It decides, for example, what framework to use, which software patterns are most suitable, and how to best test the code [2]. In the design process, the prototype and testing stage would involve software engineering. But software engineering alone lacks the data to make those decisions.  

Requirements engineering develops the scope of the project by understanding the target audiences’ needs, motivation, and goals. Requirements engineering can be categorized into four key tasks [3]:  

\begin{itemize}
  \item \textbf{Feasibility study  }, The feasibility study analyses, for example, if the project is technically feasible, economically profitable, and if the developers have enough resources like time and staff to develop the project. 
  \item \textbf{Requirements elicitation and analysis  }, The elicitation step focuses on getting to know the stakeholders; a group of people that have an interest or share in the project [4]; and figuring out the needs and expectations of the end-user. But elicitation also includes gathering data about technical requirements, for example, standards in the industry and similar projects. Finally, the data gets analyzed and reduced to the most essential requirements. 
  \item \textbf{Requirements specification  }, Requirement specification develops functional and formal requirement models from the data gathered in the previous step. Such models include functional requirements; what can the system do, and non-functional requirements; how well should the system be able to do them.  
  \item \textbf{Requirements validation    }, The final task is to validate the requirements. A valid requirement would, for example, be consistent and not conflict with any other requirement, or it needs to be practically achievable.  
\end{itemize}

\subsection{The Role of Personas and Demographic in Requirement Engineering }
Requirement engineering needs a lot of data to develop the scope of the project and specify the requirements. Especially the requirements elicitation and analysis stage requires a high involvement of humans [5]. Developers are prone to make assumptions about what the user needs and expectations; therefore, it is important to maintain regular interactions with the stakeholders [5]. But methods to manually collect data from the users, like interviews and surveys, are tedious and take a lot of time and effort. They are also subject to a sampling bias; interviewing only a small portion of the target userbase leads to a bias representation of viewpoints [5].  
