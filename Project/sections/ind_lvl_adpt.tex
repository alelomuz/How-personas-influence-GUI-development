\subsection{What is Individual-Level Adaptation?}
% Individual-level adaptation definition
While population-level adaptation applies to all users within a population, individual-level adaptation refers to GUI adaptations tailored to a single user. Such adaptations can occur either automatically, imposed by the system (\textbf{Personalization}), or be explicitly enabled by the user (\textbf{Customization}). 

% User-driven personalization is less data-driven than System-driven adaptation
The decision to implement any form of customization is typically driven by demographic and persona heterogeneity and therefore constitutes a relevant development concern. However, since this form of customization is optional and enhances the UI/UX only when actively discovered and used by the relevant user groups, it does not require the same level of demographic and persona analysis as personalization, which addresses more relevant UI enhancements that are applied automatically by the system.

\subsection{When to use Customization?}
% Personalization is preferred, but requires much data
Assuming a company has a clear and reliable understanding of the needs of a specific niche user group, the optimal approach is to implement personalization automatically within the UI/UX rather than relegating such improvements to deeply buried configuration settings that rely on users actively discovering and actively enabling them. However, given the substantial cost of false positives in such cases, implementing personalization requires strongly data-backed inferences, which can rarely be guaranteed with enough certainty. For this reason, customization is widely used in the industry, especially in applications that aim to satisfy diverse user needs.

% Less customization as possible in Mobile Apps
Mobile applications, for example, tend to offer less customization than other platforms and place a stronger emphasis on simplicity and task-oriented UX. Users of smartphone apps are typically more focused on accomplishing the actual tasks, rather than on customizing the app itself, a tendency that is partly influenced by limited screen size. \cite{apostol2024ux}

% Apps that force the user to customize
However, in some cases — such as the Yazio calorie-tracking app — the mobile application depends heavily on the user persona and must therefore require users to complete an initial customization process, such as a multiple-choice quiz.


A strong example for user user-driven customization is the Samsung's Good Lock app, which allows users of Samsung Galaxy Android devices to customize the UI/UX beyond standard system settings by extending Samsung’s One UI — for example, through lock screens, navigation bars, themes, and gesture controls.Samsung’s approach directly addresses power user personas seeking advanced customization and device optimization, without altering the default experience for most users. \cite{goodlock_wikipedia} \cite{goodlock_playstore} \cite{androidcentral_goodlock_2025}

\subsection{Personalization}

The concept of personalization is significantly broader and more complex than customization, typically extending far beyond superficial adjustments and relying heavily on demographic and persona insights. Personalization refers to the system-driven, automatic tailoring of a system’s content, layout, or behavior based on individual user data, and can be realized across many different levels and forms. \cite{personalization_wikipedia} \cite{nngroup_customization_personalization_2016}

\subsection{Recommender Systems}
Algorithmic curation represents one of the most prevalent personalization approaches, leveraging user data to infer individual preferences and tailor content, recommendations, or experiences accordingly. \cite{algorithmic_curation_wikipedia}

Of particular interest for GUI development in this context are recommender systems. Applications across several domains, including social media (e.g., Instagram, TikTok), entertainment platforms (e.g., Netflix, Spotify, YouTube), and e-commerce (e.g., Amazon), adopt homepages that are heavily driven by algorithms inferring user interests. The entire GUI development must therefore be heavily adapted to support a highly dynamic interface that operates in close conjunction with systems performing in-depth analysis of demographic and persona data. \cite{recommender_system_wikipedia} \cite{altexsoft_recommender_systems_personalization_2021}