\subsection{What is Individual-Level Adaptation?}
% Individual-level adaptation definition
While population-level adaptation applies to all users within a population, individual-level adaptation refers to GUI adaptations tailored to a single user. Such adaptations can occur either automatically, imposed by the system (\textbf{System-driven adaptation}), or be explicitly enabled by the user (\textbf{User-driven personalization}). 

% User-driven personalization is less data-driven than System-driven adaptation
The decision to implement any form of user-driven personalization is typically driven by demographic and persona heterogeneity and therefore constitutes a relevant development concern. However, since this form of personalization is optional and enhances the UI/UX only when actively discovered and used by the relevant user groups, it does not require the same level of demographic and persona analysis as system-driven adaptation, which addresses more relevant UI enhancements that are applied automatically.














% User-driven personalization
% Designed and implemented because personas and demographics indicate heterogeneity

% A development decision: exposing configurability instead of enforcing one UI

% demographics/personas influence development, even though the final choice is manual.

% Persona diversity

% Demographic variability

% Development rationale

% Configurability as a design-for-diversity decision

% Development implications (settings architecture, state handling)
% configuration







% System-driven adaptation

% Demographics/personas → logic → automatic UI behavior

% A/B testing, adaptive menus, inferred preferences

% Runtime personalization based on user data

% Data-driven

% Automatic

% Runtime or build-time