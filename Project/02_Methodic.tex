\section{Review Methodology}
\label{sec:methodology}

This paper presents a structured literature review rather than an empirical user study. The goal of the review is to synthesize how existing research addresses (i) the use of user personas in graphical user interface (GUI) development and (ii) the role of demographic factors in GUI and user experience (UX) design. To ensure transparency and reproducibility, we follow a simple but systematic process consisting of four main steps: search, selection, data extraction, and qualitative synthesis.

\subsection{Search Strategy}

We conducted a database search in major digital libraries commonly used in human--computer interaction and software engineering research, namely IEEE Xplore, the ACM Digital Library, and SpringerLink. In addition, we performed a complementary search on Google Scholar to capture relevant work that may not be indexed uniformly across these venues.

The search focused on combinations of terms related to personas, demographics, and user interfaces. Example search strings included:

\begin{itemize}
  \item \textit{("user persona" OR "personas") AND ("user interface" OR "GUI" OR "UX")}
  \item \textit{demographics AND ("usability" OR "user experience" OR "user interface")}
  \item \textit{persona-based design AND GUI}
\end{itemize}

We restricted our search to peer-reviewed publications written in English and published between 2000 and 2025. The lower bound reflects the period in which personas became widely discussed in interaction design and software engineering, while the upper bound corresponds to the time of conducting this review.

\subsection{Inclusion and Exclusion Criteria}

To keep the scope focused on GUI-relevant work, we applied the following inclusion criteria:

\begin{itemize}
  \item The paper is peer-reviewed (journal, conference, or workshop).
  \item The work addresses user interfaces, graphical user interfaces, or interactive systems with a clear human-facing component.
  \item The paper either (a) uses or discusses user personas in the context of design or evaluation, or (b) explicitly considers demographic characteristics (e.g., age, gender, expertise, cultural background) in relation to usability or UX.
\end{itemize}

We excluded papers that met one or more of the following conditions:

\begin{itemize}
  \item No clear relation to user interfaces or human--computer interaction (e.g., purely backend or infrastructure-oriented work).
  \item Marketing and business-oriented persona usage without any link to interface design or UX.
  \item Non-scientific sources such as blog posts, white papers without peer review, or theses not published in curated venues.
\end{itemize}

These criteria are intentionally simple but sufficient to filter out work that is clearly outside the scope of GUI development, while retaining a broad view on personas and demographics.

\subsection{Study Selection Process}

The database queries returned an initial set of publications. We first removed duplicates across the different digital libraries. Next, we conducted a title and abstract screening to exclude obviously irrelevant papers (e.g., pure marketing analytics or demographic surveys without any interface component).

For the remaining candidates, we inspected the full text to verify that the work (i) addressed user interfaces or interactive systems and (ii) either applied personas or analyzed demographic factors in a way that is relevant to GUI or UX design. This process resulted in a final set of studies that form the basis of our review. While we do not claim full exhaustiveness, the selection procedure aims to cover the central streams of research at the intersection of personas, demographics, and GUI development.

\subsection{Data Extraction and Synthesis}

For each included paper, we extracted a small set of descriptive and analytical attributes:

\begin{itemize}
  \item bibliographic information (authors, year, venue);
  \item application domain (e.g., web, mobile, games, enterprise systems);
  \item role of personas (e.g., requirements elicitation, design rationale, communication, evaluation);
  \item demographic factors considered (e.g., age, gender, expertise, culture, accessibility needs);
  \item reported outcomes related to usability, user experience, or user performance;
  \item study type (e.g., controlled experiment, case study, survey, conceptual or framework paper).
\end{itemize}

Rather than performing a quantitative meta-analysis, we conducted a qualitative, thematic synthesis. Specifically, we grouped papers according to recurring themes such as how personas are conceptualized and used in practice, which demographic variables are most frequently considered, and how these aspects are reported to influence GUI design decisions and UX outcomes. The thematic categories emerging from this synthesis form the structure of the findings presented in Section~\ref{sec:findings}.
